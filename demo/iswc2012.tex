\documentclass[runningheads,a4paper]{llncs}

\usepackage[utf8]{inputenc}
\usepackage[T1]{fontenc}

\usepackage[activate=compatibility]{microtype}

% autoref command
\usepackage[pdftex,urlcolor=black,colorlinks=true,linkcolor=black,citecolor=black]{hyperref}
\def\sectionautorefname{Section}
\def\subsectionautorefname{Subsection}
\def\figureautorefname{Fig.}

% todo macro
\usepackage{color}
\newcommand{\todo}[1]{\noindent\textcolor{red}{{\bf \{TODO}: #1{\bf \}}}}

\hyphenation{Isabella}

\usepackage{comment}

\begin{document}

\title{Adding Realtime Coverage to\\the Google Knowledge Graph}
\titlerunning{Adding Realtime Coverage to the Google Knowledge Graph}
\authorrunning{Adding Realtime Coverage to the Google Knowledge Graph}

\author{Thomas Steiner\inst{1}\thanks{Full disclosure: the author is also a~Google employee.} \and
		Ruben Verborgh\inst{2} \and \\
		Joaquim Gabarro\inst{1} \and 
		Rik Van de Walle\inst{2}		
}

\institute{Universitat Politècnica de Catalunya -- Department {\sc lsi}\\
		   08034 Barcelona, Spain\\
		   \urldef{\emails}\UrlFont\path|{tsteiner,gabarro}@lsi.upc.edu|
		   \emails\rm
		   \and
		   Ghent University -- IBBT,
		   ELIS -- Multimedia Lab\\
		   Gaston Crommenlaan 8 bus 201,
		   B-9050 Ledeberg-Ghent, Belgium\\
		   \urldef{\emails}\UrlFont\path|{ruben.verborgh,rik.vandewalle}@ugent.be|
		   \emails\rm		   
}

\maketitle
% Start footnotes again by 0.
\setcounter{footnote}{0}

\begin{abstract}
In May 2012, the Web search engine Google has introduced the so-called Knowledge Graph,
a~graph that understands real-world entities and their relationships to one another.
Entities covered by the Knowledge Graph include landmarks, celebrities, cities, sports
teams, buildings, movies, celestial objects, works of art, and more.
The graph enhances Google search in three main ways:
by disambiguation of search queries,
by search log-based summarization of key facts,
and by explorative search suggestions.
With this paper, we suggest a~fourth way of enhancing Web search:
through the addition of realtime coverage
of what people say about real-world entities on social networks.
We report on a~browser extension that seamlessly adds relevant microposts
from the social networking sites Google+, Facebook, and Twitter
in form of a~panel to Knowledge Graph entities.
In a~true Linked Data fashion, we interlink detected concepts in microposts
with Freebase entities, and evaluate our approach for both relevancy and usefulness.
The extension is freely available,
we invite the reader to reconstruct the examples of this paper
to see how realtime opinions may have changed since time of writing.
\end{abstract}

\section{Introduction}
With the introduction of the Knowledge Graph, the search engine Google
has made a~significant paradigm shift towards \textit{``things, not strings''}~\cite{singhal2012},
as a~post on the official Google blog states.
With the graph, Google now publicly acknowledges efforts in the direction of 
understanding the difference between queries like ``giants''
for the football team (New York Giants)
and ``giants'' for the baseball team (San Francisco Giants).
Users now get word sense disambiguation support that allows them
to steer the search engine in the right direction.

Google News is a~news aggregator portal provided and operated by Google.
Earlier this year, a~new feature called realtime coverage
was added to the portal~\cite{zuccarino2012}, providing signed in US Google+ users
with the latest relevant posts from the Google+ community for news stories.

In this paper, we have implemented a~social network agnostic approach
to add realtime coverage to Knowledge Graph results via a~browser extension.
The extension is freely available on the Chrome Web
Store\footnote{Knowledge Graph Socializer extension (\todo{Add Chrome Web Store URL})},
an also free Google API key is required.

\section{Methodology}
Chrome extensions are small software programs that users can install
to enrich their browsing experience.
Via so-called content scripts, extensions can inject and modify the contents of Web pages.
We have implemented an extension that gets activated when a~user uses Google to search the Web.
If the extension detects that the current search engine results page
has an associated real-world entity in the Knowledge Graph,
first, it extracts the entity's name and Wikipedia URL.
Second, the extension performs full-text searches via the search APIs of
the popular social networking sites Google+, Twitter, and Facebook
and returns the top $n$ results (configurable) of each network.
Third, all microposts get analyzed and annotated via part-of-speech tagging.
For each (pair of) nouns, the Freebase search API is used
to link these strings with entities (in Freebase),
a~task commonly known as entity linking~\cite{spitkovsky2012}.
We follow a~context free most frequent sense approach,
which, according to Agirre and Edmonds~\cite{agirre2007},
serves as a~surprisingly strong baseline,
an observation that we can confirm in our evaluation of micropost \emph{relevancy}
in \autoref{sec:relevancy}.
In a~final step, the resulting set of microposts is ordered chronologically
and attached in form of a~separate panel to the Knowledge Graph result.
A~screenshot with exemplary extension output for the entity
\emph{``Isabella Stewart Gardner Museum''} in Boston, MA,
is online\footnote{Screenshot of the extension (\url{http://twitpic.com/a8zgiq/full})}. 

\section{Evaluation}
For the evaluation we have considered 100 microposts from 94 unique users
out of which 72 microposts contained outbound links to overall 94 Web pages.
We have evaluated the extension on 21 real-world entities starting from the concept
\emph{``Boston,~MA''}, and then recursively following graph links to related concepts.
Our evaluation criteria were \emph{usefulness} of the information
contained in both the microposts and the potentially linked-to Web pages,
and the \emph{relevancy} of the microposts to the real-world entities in question.
We have calculated the Mean Opinion Score (MOS)~\cite{mos1998} for both criteria.

Traditionally, MOS is used for conducting subjective evaluations
of telephony network transmission quality,
however, more recently, MOS has also found wider usage in the multimedia community
for evaluating \emph{per se} subjective things
like perceived quality from the users' perspective. 
Therefore, a~set of standard, subjective tests are conducted,
where a~number of users rate the quality of test samples
with scores ranging from 1 (worst) to 5 (best).
The actual MOS is then the arithmetic mean of all individual scores.
Given our intrinsically subjective evaluation criteria,
namely \emph{usefulness} and \emph{relevancy},
MOS provides a~meaningful way to judge the overall quality of our approach.

For \emph{relevancy}, the MOS was 4.38 out of 5.
For \emph{usefulness}, we have calculated a~MOS of 3.75 out of 5.
Our complete evaluation is available online\footnote{Evaluation (\url{http://goo.gl/dbvr4})}.
In the following, we provide an interpretation of the MOS results for both criteria.

\paragraph{Usefulness:}
The MOS of 3.75 suggests potential for improvement,
albeit the information from the microposts and linked-to Web pages
was overall still considered useful.
Positively rated revealed insights were, among others, recommended restaurants,
suggested things to do, scheduled future or past events,
special (not necessarily advertisement-like) offers, user-generated photos,
news articles, travel tips, or stories of everyday life.
On closer inspection, the microposts that teared down the \emph{usefulness} MOS
were in the majority of cases still rated relatively high for \emph{relevancy}.
We could track down the relevant but not useful microposts to three categories:
(i) long microposts (Google+, Facebook) that mentioned the concept somewhere,
but that were too long to skim,
(ii) so-called
@Replies\footnote{``What are @Replies and Mentions?'' (\url{http://goo.gl/Ge2RG})}
on Twitter that are messages to other Twitter users,
but that lack the context of the conversation, and finally
(iii) so-called native check-in messages on Google+,
together with shared Foursquare check-in messages on Twitter and Facebook
via connected user profiles\footnote{``Connecting/sharing to Facebook and Twitter'' (\url{http://goo.gl/NBuCr})}
that are geotagged (and thus relevant), however,
provide no other information besides the fact that a~user was at a~certain place.
By disregarding all three types of microposts, the MOS could be increased to 3.94.

\paragraph{Relevancy:} \label{sec:relevancy}
The high MOS of 4.38 shows that the microposts
were in the majority of cases considered very relevant.
On the one hand, this is due to the well-chosen titles of concepts in the graph,
which oftentimes are either unique enough (\emph{e.g.}, \emph{``\underline{Faneuil} Hall''}),
or on the other hand, include some sort of disambiguation aid
(\emph{e.g.}, \emph{``Museum of Science, \underline{Boston}''}).
In~\cite{spitkovsky2012}, Spitkovsky and Chang have collected
an extensive set of link titles for Wikipedia concepts
and shown that an entirely context-free approach
to linking strings with concepts does consistently well.
Together with the observation above, this justifies
our concept title-based full-text search approach on social networking sites,
reflected by the MOS.

\section{Future Work and Conclusion}
Each concept in the Knowledge Graph has a~unique identifier~\cite{thalhammer2012}.
Exploiting this fact and considering that Knowledge Graph results
have already been interlinked via \texttt{owl:sameAs}
with Freebase entities~\cite{glaser2012},
we can imagine a~comments archive of things people said about real-world entities.
The SIOC Ontology~\cite{breslin2005} by Breslin \emph{et al.}
provides an ideal vocabulary for such archive,
as both the context of the comments and their provenance can be described.
Taking it one step further, we can think of not only archiving user comments,
but also adding meaning to them~\cite{steiner2013},
which would allow for interesting use cases
like searching for places that are frequented by celebrities.

Concluding, in this paper we have shown how realtime coverage
can be added to Knowledge Graph results via a~browser extension.
We have evaluated both the \emph{usefulness} and \emph{relevancy}
of the returned microposts and linked-to Web pages,
and have given an outlook on how via a~potentially enriched comments archive
about real-world entities, further use cases can be covered.
Through this work, we have demonstrated how the rather static, structured world
of real-world facts from the Knowledge Graph
(with facts like foundation date, opening hours, spouse, etc.),
can be successfully combined for fun and profit
with the rather dynamic, \emph{a~priori} unstructured world
of social network microposts.

\section*{Acknowledgments}
T. Steiner is partially supported by the European Commission
under Grant No.~248296 FP7 (\mbox{I-SEARCH} project).
R. Verborgh is funded by Ghent University,
the Interdisciplinary Institute for Broadband Technology~(\mbox{IBBT}),
the Institute for the Promotion of Innovation by Science and Technology in Flanders~(\mbox{IWT}),
the Fund for Scientific Research Flanders~(\mbox{FWO} Flanders), and the European Union.
J. Gabarr\'o is partially supported by \mbox{SGR}~2009--2015 (\mbox{ALCOM}) and
\mbox{TIN}--2007--66523~(\mbox{FORMALISM}).

\bibliographystyle{abbrv}
\bibliography{iswc2012}

\end{document}