\documentclass[runningheads,a4paper]{llncs}

\usepackage[utf8]{inputenc}
\usepackage[T1]{fontenc}

\usepackage[activate=compatibility]{microtype}

% autoref command
\usepackage[pdftex,urlcolor=black,colorlinks=true,linkcolor=black,citecolor=black]{hyperref}
\def\sectionautorefname{Section}
\def\subsectionautorefname{Subsection}
\def\figureautorefname{Fig.}

% todo macro
\usepackage{color}
\newcommand{\todo}[1]{\noindent\textcolor{red}{{\bf \{TODO}: #1{\bf \}}}}

\hyphenation{Isabella}

\usepackage{comment}

\begin{document}

\title{Adding Realtime Coverage to\\the Google Knowledge Graph}
\titlerunning{Adding Realtime Coverage to Google's Knowledge Graph}
\authorrunning{Adding Realtime Coverage to Google's Knowledge Graph}

\author{Thomas Steiner\inst{1}\thanks{Full disclosure: the author is also a~Google employee.} \and
		Ruben Verborgh\inst{2} \and \\
		Joaquim Gabarro\inst{1} \and 
		Rik Van de Walle\inst{2}		
}

\institute{Universitat Politècnica de Catalunya -- Department {\sc lsi}\\
		   08034 Barcelona, Spain\\
		   \urldef{\emails}\UrlFont\path|{tsteiner,gabarro}@lsi.upc.edu|
		   \emails\rm
		   \and
		   Ghent University -- IBBT,
		   ELIS -- Multimedia Lab\\
		   Gaston Crommenlaan 8 bus 201,
		   B-9050 Ledeberg-Ghent, Belgium\\
		   \urldef{\emails}\UrlFont\path|{ruben.verborgh,rik.vandewalle}@ugent.be|
		   \emails\rm		   
}

\maketitle
% Start footnotes again by 0.
\setcounter{footnote}{0}

\begin{abstract}
In May 2012, the Web search engine Google introduced the so-called Knowledge Graph,
a~graph that understands real-world entities and their relationships to one another.
Entities covered by the Knowledge Graph include landmarks, celebrities, cities, sports
teams, buildings, geographical features, movies, celestial objects, works of art, and more.
The graph enhances Google search in three main ways:
by disambiguation of search queries,
by search log-based summarization of key facts,
and by explorative search suggestions.
With this paper, we suggest a~fourth way of enhancing Web search:
through the addition of realtime coverage
of what people say about real-world entities on social networks.
We report on a~browser extension that seamlessly adds relevant microposts
from the social networking sites Google+, Facebook, and Twitter
as a~panel to Knowledge Graph entities.
In a~true Linked Data fashion, we interlink detected concepts in microposts
with Freebase entities, and evaluate our approach for both relevancy and usefulness.
% RV: Maybe the abstract is not the right place for URLs.
%     Perhaps something about how it's freely available and how we demonstrate examples in the paper?
The extension developed in the context of this paper was published
on the Chrome Web Store\footnote{\todo{Add Chrome Web Store URL}}.
A~screenshot with exemplary extension output for the entity
\emph{Isabella Stewart Gardner Museum} in Boston, MA,
can be found online\footnote{\url{http://twitpic.com/a8zgiq/full}}. 
\end{abstract}

\section{Introduction}

\section{Methodology}

\section{Evaluation}
We have evaluated the extension on 21 real-world entities starting from the concept
\emph{Boston,~MA}, and then recursively following graph links to related concepts.
Overall, we have considered 100 microposts from 94 unique users
out of which 72 microposts contained outbound links to overall 94 Web pages.
Our evaluation criteria were \emph{usefulness} of the information
contained in both the microposts and the potentially linked-to Web pages,
and the \emph{relevancy} of the microposts for the real-world entities in question.
We have calculated Mean Opinion Scores (MOS) for both criteria.
% RV: A reference to MOS here? Unfortunately, I only found http://www.itu.int/rec/T-REC-P.800.1-200303-S/en.
% RV: If there's space, maybe a justification why we're using MOS.
For \emph{relevancy}, the MOS was 4.38.
% RV: For clarity, ``4.38 out of 5'' (if it's correc to say it that way)?
For \emph{usefulness}, we have calculated a~MOS of 3.75.
Our complete evaluation is available online\footnote{\url{http://goo.gl/dbvr4}}.
% RV: Should we anonymize?
In the following, we provide an interpretation of the MOS results for both criteria.

\paragraph{Usefulness:}
The MOS value of 3.75 suggests potential for improvement,
albeit in general the information from the microposts and linked-to Web pages
was overall still considered useful.
Positively rated revealed insights were, among others, recommended restaurants,
suggested things to do, scheduled future or past events,
special (not necessarily advertisement-like) offers, user-generated photos,
news articles, travel tips, or stories of everyday life.
On closer inspection, the microposts that teared down the \emph{usefulness} MOS value
were in the majority of cases still rated relatively high for \emph{relevancy}.
We could track down the relevant but not useful microposts to three categories:
(i) long microposts (Google+, Facebook) that mentioned the concept somewhere,
but that were too long to skim,
(ii) so-called
@replies\footnote{\url{http://goo.gl/Ge2RG}}
% RV: Should we go for the full URL here, which is more descriptive?
%     Alternatively, we could ``What are @Replies and Mentions?'' (\url{http://goo.gl/Ge2RG}})
on Twitter that are messages to other Twitter users,
but that lack the context of the conversation, and finally
(iii) so-called check-in messages on Google+, and both Twitter and Facebook
via connected Foursquare profiles\footnote{\url{http://goo.gl/NBuCr}},
which are geotagged (and thus relevant), however,
provide no other information besides the fact that a~user was at a~certain place.
By disregarding all three types of microposts, the MOS could be increased to 3.94.

\paragraph{Relevancy:}
The high MOS value of 4.38 shows that the microposts
were in the majority of cases considered very relevant.
On the one hand, this is due to the well-chosen titles of concepts in the graph,
which oftentimes are either unique enough (\emph{e.g.}, \emph{\underline{Faneuil} Hall}),
or, on the other hand, include some sort of disambiguation aid
(\emph{e.g.}, \emph{Museum of Science, \underline{Boston}}).
In~\cite{spitkovsky2012}, Spitkovsky and Chang have collected
an extensive set of link titles for Wikipedia concepts
and show that an entirely context-free approach
to linking strings with concepts does consistently well.
Together with the observation above, this justifies
% RV: Should we need the SNS abbreviation, we can introduce it earlier.
the concept-title-based full-text search approach on Social Networking Sites,
reflected by the MOS value.


% http://lists.w3.org/Archives/Public/semantic-web/2012Jun/0028.html
%http://sameas.org/ggraph/?key=H4sIAAAAAAAAAONgVuLQz9U3MKs0LgIAXXSnTQwAAAA
%http://sameas.org/ggraph/?ggid=06y3r
%https://www.google.com/search?hl=en&sa=X&q=bill+gates&stick=H4sIAAAAAAAAAONgVuLQz9U3MKs0LgIAXXSnTQwAAAA&qscrl=1

\bibliographystyle{abbrv}
\bibliography{iswc2012}

\end{document}
