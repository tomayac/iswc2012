\documentclass[runningheads,a4paper]{llncs}

\usepackage[utf8]{inputenc}
\usepackage[T1]{fontenc}

\usepackage[activate=compatibility]{microtype}

% autoref command
\usepackage[pdftex,urlcolor=black,colorlinks=true,linkcolor=black,citecolor=black]{hyperref}
\def\sectionautorefname{Section}
\def\subsectionautorefname{Subsection}
\def\figureautorefname{Fig.}

% todo macro
\usepackage{color}
\newcommand{\todo}[1]{\noindent\textcolor{red}{{\bf \{TODO}: #1{\bf \}}}}

\usepackage{xspace}
\newcommand{\googleplus}{Google\nolinebreak\hspace{0em}\raisebox{.28ex}{\tiny\bf +}\kern-0.2ex\xspace}

\hyphenation{}

\usepackage{comment}
% \linespread{0.96}

\begin{document}

\title{SEKI@home, or Crowdsourcing\\ a~Google Knowledge Graph API}
\titlerunning{SEKI@home, or Crowdsourcing a~Google Knowledge Graph API}
\authorrunning{T. Steiner and S. Mirea}

\author{
  Thomas Steiner\inst{1}\thanks{Full disclosure: T. Steiner is also a~Google employee, S. Mirea a~Google intern.} \and
  Stefan Mirea\inst{2}
}

\institute{Universitat Politècnica de Catalunya -- Department LSI,
		   Barcelona, Spain\\
		   \urldef{\emails}\UrlFont\path|tsteiner@lsi.upc.edu|
		   \emails\rm \and
		   Computer Science, Jacobs University Bremen, Germany\\
		   \urldef{\emails}\UrlFont\path|s.mirea@jacobs-university.de|
		   \emails\rm
}

\maketitle
% Start footnotes again by 0.
\setcounter{footnote}{0}

\begin{abstract}
In May 2012, the Web search engine Google has introduced the so-called Knowledge Graph,
a~graph that understands real-world entities and their relationships to one another.
It currently contains more than 500 million objects,
as well as more than 3.5 billion facts about
and relationships between these different objects.
Soon after its announcement, people started to ask
for a~Knowledge Graph Application Programming Interface (API),
however, as of today, Google does not provide one.
With \emph{SEKI@home}, which stands for \emph{Search for Embedded Knowledge Items},
we propose a~browser extension-based approach to crowdsource such an API.
As people with the extension installed search on Google.com,
the extension sends anonymous graph facts from Search Engine Results Pages (SERPs)
to a~centralized, publicly accessible triple store,
and thus over time recreates an open Knowledge Graph.
We have implemented and made available a~prototype browser extension
tailored to the Google Knowledge Graph, however,
note that the concept of \emph{SEKI@home} is generalizable for other closed knowledge bases.
\end{abstract}

\section{Introduction}
With the introduction of the Knowledge Graph, the search engine Google
has made a~significant paradigm shift towards \textit{``things, not strings''}~\cite{singhal2012},
as a~post on the official Google blog states.
Entities covered by the Knowledge Graph include landmarks, celebrities, cities, sports
teams, buildings, movies, celestial objects, works of art, and more.
The graph enhances Google search in three main ways:
by disambiguation of search queries,
by search log-based summarization of key facts,
and by explorative search suggestions.
This triggered demand for a~Knowledge Graph API that would allow for
programmatically accessing the facts stored in the graph~\cite{quora2012}.
At time of writing, however, no such API is available.


\section{Methodology}

\section{Evaluation}

\section{Future Work and Conclusion}

\section*{Acknowledgments}
\small
T. Steiner is partially supported by the European Commission
under Grant No.~248296 FP7 (\mbox{I-SEARCH} project).

\linespread{1}
\bibliographystyle{abbrv}
\bibliography{iswc2012}

\end{document}